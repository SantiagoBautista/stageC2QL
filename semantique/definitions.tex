\documentclass[french]{article}
\usepackage[T1]{fontenc}
\usepackage[utf8]{inputenc}
\usepackage{lmodern}
\usepackage[a4paper]{geometry}
\usepackage{babel}
\usepackage{amsmath}
\usepackage{amssymb}

\newcommand{\uconcat}{\ensuremath{+\!\!\!+\,}}

\DeclareMathOperator{\proj}{\pi}
\DeclareMathOperator{\sel}{\sigma}
\DeclareMathOperator{\frag}{frag}
\DeclareMathOperator{\defrag}{defrag}
\DeclareMathOperator{\crypt}{crypt}
\DeclareMathOperator{\decrypt}{decrypt}
\DeclareMathOperator{\group}{group}
\DeclareMathOperator{\id}{id}
\DeclareMathOperator{\dom}{dom}
\DeclareMathOperator{\ens}{E}
\DeclareMathOperator{\R}{R}
\DeclareMathOperator{\Sc}{S}
\DeclareMathOperator{\s}{sch}

\newcommand\typeT[1]{\text{\ttfamily #1}}
\newcommand{\decryptArgs}[2]{\decrypt_{#1 , \typeT{#2}}}
\newcommand{\cryptArgs}[2]{\crypt_{#1 , \typeT{#2}}}
\newcommand{\projDelta}{\proj_{\delta}}
\newcommand{\selP}{\sel_p}
\newcommand{\decryptCAlpha}{\decryptArgs{\alpha}{c}}
\newcommand{\cryptCAlpha}{\cryptArgs{\alpha}{c}}
\newcommand{\ch}{\typeT{c}}
\newcommand{\chp}{\typeT{c'}}
\newcommand{\groupDelta}{\group_{\delta}}
\newcommand{\fragDelta}{\frag_{\delta}}
\newcommand{\val}{\mathcal{V}}

\newtheorem{defi}{Définition}

\begin{document}
Le but de ce document est de donner une définition
formelle des fonctions dont est composé le langage C2QL.

\subsection*{Définitions générales}
Soit $\val$ un ensemble, appelé ensemble des valeurs.
\begin{defi}
	Ici, pour simplifier, on appelle chaîne de caractères
	tout mot sur l'alphabet 
	$$
	\Sigma = \{a, \dots, z\} \cup \{A, \dots, Z \} \cup \{0, \dots,  9 \}
 	$$
\end{defi}

\begin{defi}
	On appelle \emph{nom d'attribut} toute chaîne de caractères.
\end{defi}	

\begin{defi}
	On appelle \emph{schéma relationnel} tout ensemble
	de noms d'attributs.
\end{defi}

\begin{defi}
	On appelle \emph{relation} de schéma relationnel $\Delta$
	un ensemble de fonctions de $\Delta \cup \{ id \}$ dans $\val$.
	
	Chacune de ces fonctions (chacun des éléments de la relation)
	est appelé(e) \emph{ligne}.
	
	Pour chaque ligne $l$ de la relation et chaque $\alpha$ de $\Delta$,
	$l(\alpha)$ est appelé \emph{attribut} de \emph{nom} $\alpha$ pour la ligne $l$.
	
	L'image de $id$ est appelé \emph{identifiant de la ligne}, et il est, au sein de chaque
	relation, unique pour chaque ligne.
\end{defi}

\begin{defi}
	On appelle $\Sc$ l'ensemble des schémas relationnels possibles.
	Autrement dit, on pose  $\Sc = \left( \Sigma^* \right)^*$.
	
	On appelle $\R$ l'ensemble des relations possibles,
	
	et on introduit la fonction $\s$ de $\R$ dans
	$\Sc$
	qui à une relation associe son schéma relationnel.
\end{defi}

\subsection*{Projection et sélection}
\begin{defi}
	Pour tout ensemble $\delta$ de noms d'attributs,
	on appelle \emph{projection sur les attributs $\delta$}
	la fonction suivante:
	$$
	\begin{array}{llcl}
	\projDelta 	& \R 	& \rightarrow 	& \R \\
				& r		& \mapsto		& 
					\{{l|}_{(\delta\cap \s(r)) \cup \{id\}} / l \in r \}
	\end{array}
	$$
	
\end{defi}


\end{document}
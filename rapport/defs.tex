Les définitions des différents opérateurs présentes dans la thèse de 2016
sont données en français, ce qui facilite leur compréhension, mais rend
impossible une preuve mathématique de la correction des lois les concernant.

J'ai donc commencé par poser des définitions formelles des fonctions du langage
C2QL.

Pour former ces définitions, j'ai du faire des choix à plusieurs reprises.
À chaque fois, j'ai utilisé les deux mêmes critères pour guider mes choix:
d'une part il faut que les définitions que je choisi permettent au langage
d'être le plus expressif possible, d'autre part il faut que les définitions
choisies facilitent une démonstration rigoureuse de la correction
des lois de commutation.

\subsubsection*{Des tuples ou des fonctions?}
Dans son livre de 1982, Ullman parle de deux
façons de définir les relations (tables) de l'algèbre relationnelle:
soit comme un sous-ensemble d'un produit de domaines
(donc comme un ensemble de tuples, où chaque
tuple représente une ligne
de la table et chaque coordonnée de chaque tuple correspond
à la valeur pour un attribut donné), soit comme un ensemble de fonctions
définies sur l'ensemble des attributs (chaque fonction correspond donc
à une ligne de la table, et la valeur d'un attribut pour une ligne donnée
est son image par cette fonction).

Ullman décide d'utiliser la première définition pour le reste de son livre,
et ne détaille pas plus la deuxième définition. C'est pourtant cette deuxième
définition que j'ai décidé de prendre dans ce cas ci, puisque que ce soit
pour le chiffrement comme pour la fragmentation, on fait ici référence aux différentes
colonnes d'une table par leur nom (toutes les colonnes ont un nom, et tous les
noms des colonnes sont différents) et il est donc plus facile de raisonner,
par exemple, sur la fragmentation, avec cette définition-là:
en effet, la fragmentation se définit alors en thermes de simples restrictions
sur certains ensembles.

\subsubsection*{Des sélections portant sur plus d'un attribut à la fois.}
Dans certaines des lois présentes dans la thèse, la notation
suggère que les prédicats utilisés lors des filtrages ne portent que sur
un seul attribut à la fois
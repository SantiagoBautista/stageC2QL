Comme vu à la section \ref{context}, l'objectif
de ce stage était de compléter et de démontrer
un ensemble de lois décrivant dans quelles situations
les fonctions qui constituent le langage C2QL commutent
entre elles.
Comme vu à la section \ref{contrib}, pour atteindre ces
deux objectifs, il a fallu d'abord 
poser un cadre sémantique plus précis 
(\emph{i.e.} donner des définitions plus formelles)
aux fonctions du langage C2QL.

Ces trois choses (poser un cadre formel plus précis,
compléter l'ensemble des propriétés et démontrer 
l'ensemble des propriétés en question) ont été
réussies pendant le stage.
Ces trois choses ayant été faites, nous avons commencé
à développer une première approche, naïve, d'un optimiseur
automatique des requêtes C2QL.

Comme exposé à la section \ref{discusion},
il reste beaucoup de choses qui pourraient
être faites pour améliorer ou compléter le langage C2QL;
comme par exemple intégrer d'autres critères et mécanismes
de sécurité, automatiser le processus 
d'optimisation des requêtes d'une façon moins
naïve que celle proposée dans ce stage, ou
encore créer un compilateur qui transforme
les programmes C2QL en applications concrètes,
capable de manipuler des données.
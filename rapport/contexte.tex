De plus en plus d'applications, en particulier
les applications web et les applications pour téléphone portable,
cherchent à utiliser le nuage, soit pour stocker du code ou des données, 
soit pour faire des calculs,
voir les deux à la fois.

En effet, le nuage peut offrir des services (que ce soit sous forme d'infrastructure,
de plateforme ou de logiciel) disposant d'une forte disponibilité et faciles à redimensionner.

Autrement dit, pouvoir utiliser le nuage est devenu un enjeu de la conception logicielle,
à cause des avantages de disponibilité et redimensionnement que cela offre.

Mais ce n'est pas le seul enjeux de la conception logicielle.

Vu que la plupart de ces applications manipulent des données personnelles, 
garantir la \emph{confidentialité} des données est également un enjeux de 
ces applications là; tout comme le sont les \emph{performances} pour garantir
une meilleure expérience à l'utilisateur de l'application.

Dans sa thèse, R. Cherrueau s'intéresse à trois techniques particulières
utilisées dans le développement logiciel et regarde comment elles interagissent
avec les trois enjeux cités plus haut.
Ces trois techniques, qu'on va décrire brièvement, sont
	le \emph{chiffrement},
	la \emph{fragmentation verticale} et
	l'\emph{exécution} de l'application \emph{chez l'utilisateur}.

\paragraph{Le chiffrement}
Lorsqu'il est bien utilisé, le chiffrement permet de garantir la confidentialité
des données de l'utilisateur.
De plus, dans certains cas, des calculs peuvent être faits sur les données chiffrées.
On appelle chiffrement homomorphe un chiffrement avec lequel on peut effectuer des calculs
avec les données chiffrées. Les chiffrement homomorphes totaux, comme celui de
Gentry (\textbf{référence à ajouter}) sont pour l'instant trop contraignants pour pouvoir être utilisés
dans la plupart des applications, mais les chiffrements homomorphes partiels,
c'est à dire les chiffrement avec lesquels on peut effectuer \emph{certaines}
opérations sur les données chiffrées, peuvent se révéler très utiles.
C'est le cas des chiffrements déterministes (dont les chiffrements symétriques)
qui sont des chiffrement homomorphes partiels, permettant le test d'égalité.

Dans tous les cas, le chiffrement implique un surcoût en terme de calculs,
donc diminue les performances. Dans certains cas, il améliore la confidentialité
et permet l'utilisation du nuage.

\paragraph{La fragmentation verticale} consiste en séparer les différentes données dont

\begin{figure}
	\begin{center}
		\includegraphics[width=0.7\textwidth]{snps.png}
		\caption{Enjeux et techniques dans le cloud-computing}
		\caption*{(image provenant de la thèse de Ronan Cherrueau)}
		\label{enjeux}
	\end{center}
\end{figure}
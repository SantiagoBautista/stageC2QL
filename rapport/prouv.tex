Pour pouvoir utiliser les lois de commutation
pour créer des programmes C2QL optimaux,
il est nécessaire de s'assurer que ces lois sont correctes,
c'est à dire que le résultat obtenu avant et après la
commutation des opérateurs est bien le même.

Pour démontrer la correction des lois de commutation,
deux approches sont possibles:
soit nous essayons de les démontrer à la main,
soit nous essayons de les démontrer de façon automatique
avec un \emph{prouveur} ou un \emph{assistant de preuve}
tel que \typeT{Coq}.

Nous avons opté pour la première approche,
principalement pour des raisons de temps.
En effet, vu qu'il s'agit d'un stage de L3
nous n'avons aucune expérience dans l'utilisation
des prouveurs ou des assistants de preuve,
et donc un temps non négligeable aurait été nécessaire
à la prise en main de tels outils.
Or, l'appropriation du sujet et du langage C2QL
avaient déjà pris un temps certain, donc
nous n'étions pas sûrs d'avoir le temps nécessaire.
Ceci est d'autant plus vrai que nous n'avions
aucun moyen de nous assurer que les méthodes 
de preuves automatiques (ou semi-automatiques) seraient
adaptés au problème en question.

\textbf{La structure des preuves} est,
à quelques exceptions près, toujours la même.
En effet, il s'agit, de montrer que pour deux fonctions
$f$ et $g$ du langage C2QL, les fonctions
$f \circ g$ et $g \circ f$ sont, selon le cas,
égales ou équivalentes.

Pour cela la méthode suivie est de considérer une relation
(ou un tuple de relations) $r$ quelconque et de montrer
que $(f \circ g) (r)$ est égal (ou équivalent)
à $(g \circ f) (r)$.

Mis à part certains cas où la fonction
$\frag$ intervient,
$(f \circ g) (r)$ et $(g \circ f) (r)$ sont tous les deux
des ensembles (finis), et donc pour prouver leur égalité
(ou leur équivalence) soit on prouve qu'ils sont
l'un inclus dans l'autre et l'autre inclus dans l'un,
soit on prouve que l'un est inclus dans l'autre et qu'ils
ont tous deux même cardinal.

\paragraph{Égalité ou équivalence}
La plupart du temps, c'est bien d'une égalité qu'il s'agit
entre les deux résultats en question.
Cependant, on va ci-dessous expliquer pourquoi
lorsque l'une des fonctions qu'on cherche à faire commuter
est la fonction de jonction, on ne peux pas parler
d'égalité proprement dite et il faut parler d'équivalence.

Cela vient de la gestion des identifiants des lignes.
Pour pouvoir défragmenter des relations issues d'une
fragmentation, on utilise des identifiants pour les lignes:
au sein de chacune des relations manipulées, chaque ligne
possède un identifiant qui (dans la relation) 
lui est unique.
Ainsi, lors de la fragmentation, les deux morceaux
d'une ligne gardent le même identifiant,
ce qui permet de les faire correspondre lors de la 
défragmentation.
Comme au sein de chaque relation chaque identifiant
est unique, cette correspondance se fait sans ambiguïté;
c'est bien le comportement désiré.

Cependant, lors d'une jointure, une ligne d'une des tables
jointes peut donner lieu à plusieurs lignes dans
le résultat, et il faut bien que ces lignes-là aient
un identifiant
(dans le cas où l'on voudrait les fragmenter par
la suite). Or cet identifiant ne peut pas correspondre
à celui de l'une des lignes qui lui a donné naissance,
parce qu'il ne serait alors pas unique.
Il ne peut pas non plus correspondre à un couple
avec les deux identifiants des deux lignes lui ayant
donné naissance, car une autre jonction
(avec un ensemble différent d'attributs en communs,
par exemple après une projection) pourrait donner
lieu, ailleurs dans le système,
à des lignes ayant le même identifiant que la ligne
crée par la jonction, sans que ces deux lignes aient
un rapport entre elles.

Ainsi, les effets propres à la jonctions nécessitent
que lors d'une jonction, les lignes du résultat 
aient des identifiants frais, qui ne soient
présents nulle part ailleurs dans le système.

Une conséquence directe de cela, est que
les lois de commutation ne peuvent pas assurer l'égalité
stricte des résultats lorsqu'une jonction intervient,
car la jonction crée un identifiant frais pour chaque
ligne résultat, et que deux identifiants frais
(ceux de chaque résultat) n'ont aucune raison d'être
égaux entre eux.

Dans ces cas-là il s'agit donc de prouver une 
égalité \emph{aux identifiants des lignes près}.
C'est ça qu'on appelle \emph{équivalence}.
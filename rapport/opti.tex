Une partie importante dans le développement des
applications C2QL est le passage d'une requête
sous sa forme locale (sans tenir compte
du fait qu'elle s'exécute dans le cloud,
ni tenir compte des mécanismes de protection utilisés)
à une version optimisée.

En effet, tout l'intérêt du langage C2QL est d'obtenir
un programme tirant avantage de la composition
pour être particulièrement performant
(au sens des trois critères définis dans la section
\ref{context}) tout en facilitant au maximum
le développement pour le développeur
(en lui permettant de n'écrire que la version locale
des requêtes).

C'est l'utilisation des lois de commutation qui permet
de passer de la version locale à la version
optimisée, mais cette utilisation se fait pour le moment à
la main: à chaque étape, le développeur doit regarder, 
dans les lois applicables, laquelle il veut appliquer.

Pour faciliter la tache au programmeur, on peut
vouloir automatiser ce processus d'optimisation,
ce qui pose plusieurs questions.

\paragraph{Choix des métriques}
Les lois, peuvent avoir des effets sur les trois
objectifs/critères suivants:
\begin{itemize}
	\item Utiliser le cloud le plus possible
	\item Utiliser le moins de temps possible
	\item Faire transiter le moins de données
	possibles à travers le réseau
\end{itemize}
Ainsi donc, si jamais on doit choisir entre deux lois de commutation différentes applicables à une requête donnée
et mutuellement exclusives; pour faire un choix objectif
il faut disposer d'une manière de mesurer l'effet produit
par chacune de ces lois par rapport à ces critères-ci.

Qui plus est, si les lois parmi lesquelles il faut choisir
agissent sur des critères différents, il
peut être nécessaire de pouvoir comparer l'importance des
différents critères entre eux. Autrement dit,
si l'on dispose d'une métrique pour chaque critère,
il faut avoir une façon d'en déduire une métrique
globale. Ceci peut nécessiter de s'intéresser aux
nécessités et visées spécifiques de l'application en question.

\paragraph{Choix de la méthode d'exploration}
Les différentes suites de lois appliquées
pour optimiser une requête se présentent comme
un arbre de transformations possibles, qu'il faut donc
explorer pour trouver la forme la plus optimale
pour la requête.

Il faut donc choisir une méthode d'exploration
pour explorer l'arbre en question.

\paragraph{Une première approche, naïve}
En ce moment, nous sommes en train de développer
une toute première approche, complètement naïve
d'un optimiseur automatique de requêtes C2QL.
La méthode d'exploration choisie est l'exploration
exhaustive. La métrique choisie consiste en,
pour chaque lois et chaque critère, affecter un score
de 1, 0 ou -1 selon le fait que la loi favorise,
soit neutre, ou défavorise le critère en question;
et on affecte la même importance, le même poids,
aux métriques des trois critères, par une simple somme,
pour obtenir la métrique globale.
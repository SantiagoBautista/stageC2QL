Comme nous l'avons vu dans la section
\ref{contrib}, le but de ce stage 
était surtout de gagner en assurance et en
\og complétude \fg{} sur un aspect précis
du langage C2QL: l'ensemble de lois
de commutation qui permettent de passer d'optimiser une
requête C2QL.

Ce but-là a été réussi, mais il reste, en ce qui
concerne le langage C2QL, des ajouts et
approfondissements intéressants qui peuvent être faits
(tels que l'ajout d'un compilateur vers une application
concrète ou la prise en compte d'autres critères de
sécurité et d'autres mécanismes de sécurité), et,
en ce qui concerne le travail de ce stage, 
l'automatisation de l'optimisation de requêtes,
qui n'a été que effleurée, pourrait être traitée
de façon plus approfondie.

\subsection*{Un compilateur concret de C2QL}
Pour l'instant, C2QL dispose d'une implémentation
en tant que langage embarqué dans Idris; qui permet
de vérifier certaines propriétés de \og bonne formulation
\fg{} des requêtes qu'on écrit en C2QL. Par exemple,
lorsque, dans une requête, on déchiffre
un attribut, l'implémentation actuelle
vérifie que l'attribut en question était bien chiffré.
Ce genre de vérifications permet d'éviter un certain
nombre d'erreurs de programmation.

Pour information, Idris est un langage avec des types
dépendants, ce qui veut dire qu'en Idris, le type
des variables peut dépendre d'une ou plusieurs valeurs.
Ainsi, on peut préciser comme type 
\og Liste avec 3 éléments \fg{} et ce type là
\emph{dépend} de la valeur 3.
Mais les valeurs dont dépend un type ne doivent pas
forcément être constantes. Le type d'une variable peut
en effet dépendre de la valeur d'une autre variable,
ou même du résultat de l'application d'un certain type
de fonction, les fonctions \emph{totales}, à une variable.
Pour faire simple, une fonction est dite \emph{totale}
en Idris, si Idris est capable de vérifier statiquement
que cette fonction termine.
Par exemple, le calcul de la longueur d'une liste peut
être implémenté de façon totale, et donc
\og un couple de listes de même longueur \fg{}
peut très bien être un type en Idris.

Grâce à l'utilisation de types dépendants en Idris,
en typant les requêtes avec les schémas relationnels
des relations utilisées,
l'implémentation actuelle de C2QL peut vérifier
de façon fiable des propriétés qui seraient autrement
plus difficiles à obtenir.

\textbf{Cependant}, même si l'implémentation
actuelle permet d'exprimer les requêtes à effectuer
et d'en vérifier des propriétés; elle n'offre aucun moyen
d'exécuter concrètement ces requêtes sur des
vraies données.
Il a été envisagé par l'auteur de la thèse
de réaliser un compilateur qui transforme un programme
C2QL (i.e. un ensemble de requêtes C2QL) en
des programmes écrits par exemple en JavaScript
pour être exécutés l'un par la machine du client,
les autres par les différents clouds concernés.
Un travail futur pourrait consister en l'élaboration
d'un tel compilateur de C2QL vers une application
\og concrète \fg{}, exécutable sur le cloud.

De la même façon,
l'auteur de la thèse avait envisagé de
créer un \og compilateur \fg{} de C2QL vers
ProVérif pour pouvoir, avec ProVérif,
vérifier que le programme C2QL écrit respecte bien
les contraintes de confidentialité qui avaient été fixées.

Il existe actuellement une transformation de
C2QL vers le pi-calcul; il resterait donc
à créer une transformation du pi-calcul vers ProVérif.

\subsection*{Un optimiseur réaliste des requêtes}
Comme mentionné dans la section \ref{opti},
l'optimiseur automatique de requêtes
que nous sommes en train de développer 
suit une approche naïve (tantôt dans sa façon de
mesurer l'effet de l'application d'une commutation, 
comme dans l'exploration des suites de commutations
possibles) et a pour seule ambition d'être une
première approche.

Un travail futur pourrait être de créer un optimiseur
de requêtes plus réaliste; ce qui nécessiterait
une réflexion plus approfondie, en particulier
en ce qui concerne les métriques permettant d'estimer
l'effet d'une commutation donnée.

\subsection*{D'autres critères et mécanismes de sécurité}
C2QL est un langage qui permet d'exprimer la composition 
de différents mécanismes de sécurité, comme on l'a vu dans
la section \ref{descrC2QL}.

Les trois mécanismes de sécurité actuellement
pris en compte dans C2QL
(le chiffrement, la fragmentation verticale
et le fait de réaliser les calcul sur
la machine du client)
sont tous les trois des mécanismes permettant
de protéger la \emph{confidentialité} des données.

Or, la confidentialité n'est pas le seul critère
de sécurité que l'on peut vouloir garantir.
L'intégrité des données ou la disponibilité des données,
par exemple, peuvent également être des critères 
de sécurité importants; et pour les satisfaire,
il est nécessaire de prendre en compte d'autres
mécanismes de protection.

Par exemple, le tatouage des données 
(qui consiste à altérer légèrement les données,
à en ajouter une \og marque \fg{}) peut être utilisé
pour protéger l'intégrité des données.
En effet, si la marque en question constitue une
signature électronique, alors toute attaque voulant porter
atteinte à l'intégrité des données (en remplaçant les 
vraies données par des fausses données) serait
repérable par une vérification de la marque
associée à ces données là.

Ajouter d'autres mécanismes de sécurité
(tels que le tatouage) pour prendre en compte
d'autres critères de sécurité (tels que l'intégrité)
serait une extension possible du langage C2QL.
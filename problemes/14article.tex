\documentclass[french]{article}
\usepackage[T1]{fontenc}
\usepackage[utf8]{inputenc}
\usepackage{lmodern}
\usepackage[a4paper]{geometry}
\usepackage{babel}
\usepackage{amsmath}
\usepackage{amssymb}
\usepackage{graphicx}

\newcommand{\uconcat}{\ensuremath{+\!\!\!+\,}}

\DeclareMathOperator{\proj}{\pi}
\DeclareMathOperator{\sel}{\sigma}
\DeclareMathOperator{\frag}{frag}
\DeclareMathOperator{\defrag}{defrag}
\DeclareMathOperator{\crypt}{crypt}
\DeclareMathOperator{\decrypt}{decrypt}
\DeclareMathOperator{\group}{group}
\DeclareMathOperator{\id}{id}
\DeclareMathOperator{\dom}{dom}

\newcommand\typeT[1]{\text{\ttfamily #1}}
\newcommand{\decryptArgs}[2]{\decrypt_{\typeT{#1}, #2}}
\newcommand{\cryptArgs}[2]{\crypt_{#1 , \typeT{#2}}}
\newcommand{\projDelta}{\proj_{\delta}}
\newcommand{\selP}{\sel_p}
\newcommand{\decryptCAlpha}{\decryptArgs{c}{a}}
\newcommand{\cryptCAlpha}{\cryptArgs{\alpha}{c}}
\newcommand{\ch}{\typeT{c}}
\newcommand{\chp}{\typeT{c'}}
\newcommand{\groupDelta}{\group_{\delta}}
\newcommand{\fragDelta}{\frag_{\delta}}


\begin{document}
En fait, l'erreur que j'énonce ci-dessous est résolue dans la thèse.
Je m'en était pas aperçu au moment de faire ce document.

\section*{Lois (14) et (15) de l'article A Language for the Composition of Privacy-Enforcement Techniques}
\subsection*{Ce qui est écrit dans l'article}
\includegraphics[width = .9\textwidth]{lois14-15.png}

\subsection*{Contre-exemple}
Le chiffrement pris pour l'exemple est artificiel, pour privilégier la simplicité de l'exemple. \\

On prend pour prédicat p 
$$p: a_1 + a_2 < 10$$ 

pour fonction de chiffrement s
$$ s: n \mapsto n + 50 $$

et pour ensemble des attributs chiffrés a
$$ a = \{a_1\} $$

Le domaine de $p$ est alors $\{a_1, a_2\}$
qui n'est pas une partie de $a$.
On est donc dans les hypothèses mentionnées
dans l'article pour la loi (14)

On s'intéresse à la relation r
\begin{center}
	\begin{tabular}{rr}
		a\(_{\text{1}}\) & a\(_{\text{2}}\)\\
		\hline
		51 & 2\\
	\end{tabular}
\end{center}

L'image de r par
$\sel_p \circ \decryptArgs{s}{a_1}$
est la relation
\begin{center}
	\begin{tabular}{rr}
		a\(_{\text{1}}\) & a\(_{\text{2}}\)\\
		\hline
		1 & 2\\
	\end{tabular}
\end{center}

L'image de r par
$\decryptArgs{s}{a_1} \circ \sel_p$
est la relation vide.

Ainsi donc, la relation (14)
dans l'article est fausse
car la condition donnée n'est pas assez restrictive.

\subsection*{Correction possible}
Ce problème est résolu si on s'intéresse à l'intersection entre $\dom(p)$
et $a$ .

\begin{align*}
\selP \circ \decryptCAlpha 
& \equiv \decryptCAlpha \circ \selP
& \text{si $\dom(p) \cap a = \emptyset$} \\
\selP \circ \decryptCAlpha 
& \equiv \decryptCAlpha \circ \sel_{\typeT{c} \Rightarrow p}
& \text{si $p$  est compatible avec $\typeT{c}$} 
\end{align*}



\end{document}

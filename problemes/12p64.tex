\documentclass[french]{article}
\usepackage[T1]{fontenc}
\usepackage[utf8]{inputenc}
\usepackage{lmodern}
\usepackage[a4paper]{geometry}
\usepackage{babel}
\usepackage{amsmath}
\usepackage{amssymb}
%\usepackage{latexsym} %maybe needed for \Join %nope, it doesn't seem so

\newcommand{\uconcat}{\ensuremath{+\!\!\!+\,}}

\DeclareMathOperator{\proj}{\pi}
\DeclareMathOperator{\sel}{\sigma}
\DeclareMathOperator{\frag}{frag}
\DeclareMathOperator{\defrag}{defrag}
\DeclareMathOperator{\crypt}{crypt}
\DeclareMathOperator{\decrypt}{decrypt}
\DeclareMathOperator{\group}{group}
\DeclareMathOperator{\id}{id}
\DeclareMathOperator{\dom}{dom}
\DeclareMathOperator{\ens}{E}
\DeclareMathOperator{\R}{R}
\DeclareMathOperator{\Sc}{S}
\DeclareMathOperator{\s}{sch}
\DeclareMathOperator{\ls}{L}
\DeclareMathOperator{\ru}{Ru}
\DeclareMathOperator{\uni}{Unif}
\DeclareMathOperator{\cor}{cor}
\DeclareMathOperator{\rj}{Rj}
\DeclareMathOperator{\enc}{Enc}
\DeclareMathOperator{\dec}{Dec}
\DeclareMathOperator{\ids}{IDs}
\DeclareMathOperator{\lgr}{lg}
\DeclareMathOperator{\redu}{red}
\DeclareMathOperator{\head}{hd}
\DeclareMathOperator{\tail}{tl}
\DeclareMathOperator{\hfrag}{hfrag}
\DeclareMathOperator{\hdefrag}{hdefrag}

\newcommand\typeT[1]{\text{\ttfamily #1}}
\newcommand{\decryptArgs}[2]{\decrypt_{#1 , \typeT{#2}}}
\newcommand{\cryptArgs}[2]{\crypt_{#1 , \typeT{#2}}}
\newcommand{\projDelta}{\proj_{\delta}}
\newcommand{\selP}{\sel_p}
\newcommand{\decryptCAlpha}{\decryptArgs{\alpha}{c}}
\newcommand{\cryptCAlpha}{\cryptArgs{\alpha}{c}}
\newcommand{\ch}{\typeT{c}}
\newcommand{\chp}{\typeT{c'}}
\newcommand{\groupDelta}{\group_{\delta}}
\newcommand{\fragDelta}{\frag_{\delta}}
\newcommand{\val}{\mathcal{V}}
\newcommand{\cy}[1]{\typeT{c}(#1)}
\newcommand{\dc}[1]{\typeT{c}^{-1}(#1)}
\newcommand{\cip}{\cup \{id\}}
\newcommand{\fold}[3]{\operatorname{fold}_{#1, #2, #3}}
\newcommand{\foldAlphafz}{\fold{\alpha}{f}{z}}
\newcommand{\dilta}{{\delta \cip}}

\begin{document}
Je me suis aperçu qu'il faut rajouter une condition à la loi (12)
de la thèse.


\subsection*{Ce qui est écrit dans la thèse}
$$
\projDelta \circ \defrag
\equiv \defrag \circ (\proj_{\delta \cap \delta'} , \proj_{\delta \setminus \delta'})
$$
où $\delta'$ est le schéma relationnel du premier argument.
\subsection*{Contre-exemple}
Si les deux tables ont des attributs en commun,
la fonction de droite peut être définie sur les deux tables
sans que la fonction de gauche le soit.

Pour exhiber un contre-exemple on va travailler avec trois attributs $a_1$, $a_2$, $a_3$
et deux tables $r_1$ et $r_2$ définies comme suit:
\begin{center}
\begin{tabular}{cc}
	$r_1$ & $r_2$ \\
	\begin{tabular}{ccc}
		$id$ & $a_1$ & $a_2$ \\
		$1$	& "a"	& $5$
	\end{tabular}
	&
	\begin{tabular}{ccc}
		$id$ & $a_3$ & $a_2$ \\
		1	& $3$		& $7$ 	
	\end{tabular}
\end{tabular}
\end{center}
et on pose $\delta = \{a_2, a_3\}$.

Dans ce cas là, on a
$\delta' = \{a_1, a_2\}$. \\

Si on commence par les projections, il est possible de défragmenter les tables obtenues.
En effet,
après projection on a:
\begin{center}
	\begin{tabular}{cc}
		$\proj_{\delta \cap \delta'}(r_1)$ & $\proj_{\delta \setminus \delta'}(r_2)$ \\
		\begin{tabular}{cc}
			$id$ & $a_2$ \\
			$1$	&  $5$
		\end{tabular}
		&
		\begin{tabular}{cc}
			$id$ & $a_3$  \\
			1	& $3$ 	
		\end{tabular}
	\end{tabular}
\end{center}
et donc
\begin{center}
	\begin{tabular}{c}
		$\defrag(\proj_{\delta \cap \delta'}(r_1), \proj_{\delta \setminus \delta'}(r_2))$ \\
		\begin{tabular}{ccc}
			$id$ & $a_2$ & $a_3$\\
			$1$	&  $5$	& 3
		\end{tabular}
	\end{tabular}
\end{center}
Ainsi, $\defrag \circ (\proj_{\delta \cap \delta'} , \proj_{\delta \setminus \delta'})$
est définie sur ($r_1$, $r_2$).

Par contre, $\defrag(r_1, r_2)$ n'est pas défini, donc
$\projDelta \circ \defrag$ n'est pas définie sur $(r_1, r_2)$.

\subsection*{Correction}
La loi devient juste si l'on rajoute une condition sur les schémas relationnels des arguments.\\

En appelant $\delta_1$ le schéma relationnel du premier
argument et $\delta_2$ le schéma relationnel du deuxième
argument, on a:
\begin{align*}
\projDelta \circ \defrag
& = \defrag \circ (\proj_{\delta \cap \delta_1}, \proj_{\delta \cap \delta_2})
& \text{si $\delta_1 \cap \delta_2 = \emptyset$}
\end{align*}

\end{document}
\documentclass[french]{article}
\usepackage[T1]{fontenc}
\usepackage[utf8]{inputenc}
\usepackage{lmodern}
\usepackage[a4paper]{geometry}
\usepackage{babel}
\usepackage{amsmath}
\usepackage{amssymb}
\usepackage{graphicx}

\newcommand{\uconcat}{\ensuremath{+\!\!\!+\,}}

\DeclareMathOperator{\proj}{\pi}
\DeclareMathOperator{\sel}{\sigma}
\DeclareMathOperator{\frag}{frag}
\DeclareMathOperator{\defrag}{defrag}
\DeclareMathOperator{\crypt}{crypt}
\DeclareMathOperator{\decrypt}{decrypt}
\DeclareMathOperator{\group}{group}
\DeclareMathOperator{\id}{id}
\DeclareMathOperator{\dom}{dom}

\newcommand\typeT[1]{\text{\ttfamily #1}}
\newcommand{\decryptArgs}[2]{\decrypt_{\typeT{#1}, #2}}
\newcommand{\cryptArgs}[2]{\crypt_{#1 , \typeT{#2}}}
\newcommand{\projDelta}{\proj_{\delta}}
\newcommand{\selP}{\sel_p}
\newcommand{\decryptCAlpha}{\decryptArgs{c}{a}}
\newcommand{\cryptCAlpha}{\cryptArgs{\alpha}{c}}
\newcommand{\ch}{\typeT{c}}
\newcommand{\chp}{\typeT{c'}}
\newcommand{\groupDelta}{\group_{\delta}}
\newcommand{\fragDelta}{\frag_{\delta}}


\begin{document}
\section*{Loi (3), page 30 de la thèse}
\subsection*{Ce qui est écrit dans la thèse}
$$ 
\proj_{a_1} \circ \dots \circ \proj_{a_n} 
\equiv \proj_{a_1, \dots, a_n}
$$

\subsection*{Contre-exemple}
Si on considère la relation r suivante
\begin{center}
	\begin{tabular}{lr}
		a\(_{\text{1}}\) & a\(_{\text{2}}\)\\
		\hline
		a & 1\\
		b & 2\\
	\end{tabular}
\end{center}
son image par \(\pi_{a_2}\) est
\begin{center}
	\begin{tabular}{r}
		a\(_{\text{2}}\)\\
		\hline
		1\\
		2\\
	\end{tabular}
\end{center}
dont l'image par \(\pi_{a_1}\) est la table vide

Ainsi, l'image de r par
\(\pi_{a_1} \circ \pi_{a_2}\) est la table vide.

Par contre, l'image de r par
\(\pi_{a_1, a_2}\)
est la table r elle-même,
qui est différente de la table vide.



\subsection*{Correction}
La composition des projections correspond à la projection sur \emph{l'intersection},
et non pas à une projection sur l'union.

$$
\proj_{\delta_1} \circ \dots \circ \proj_{\delta_n} 
\equiv \proj_{\delta_1 \cap \dots \cap \delta_n}
$$





\end{document}

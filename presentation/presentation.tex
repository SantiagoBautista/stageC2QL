\documentclass{beamer}
\usetheme{Warsaw}
\usepackage{default}

\usepackage[T1]{fontenc}%
\usepackage[utf8]{inputenc}%
\usepackage[main=francais,english]{babel}%

\usepackage{graphicx}%
\usepackage{url}%
\usepackage{amsmath}
\usepackage{mathpazo}%
\usepackage{amssymb}

\newcommand{\uconcat}{\ensuremath{+\!\!\!+\,}}

\DeclareMathOperator{\proj}{\mathchar"1119}
\DeclareMathOperator{\sel}{\mathchar"111B}
\DeclareMathOperator{\frag}{frag}
\DeclareMathOperator{\defrag}{defrag}
\DeclareMathOperator{\crypt}{crypt}
\DeclareMathOperator{\decrypt}{decrypt}
\DeclareMathOperator{\group}{group}
\DeclareMathOperator{\id}{id}
\DeclareMathOperator{\dom}{dom}
\DeclareMathOperator{\ens}{E}
\DeclareMathOperator{\R}{R}
\DeclareMathOperator{\Sc}{S}
\DeclareMathOperator{\s}{sch}
\DeclareMathOperator{\ls}{L}
\DeclareMathOperator{\ru}{Ru}
\DeclareMathOperator{\uni}{Unif}
\DeclareMathOperator{\cor}{cor}
\DeclareMathOperator{\rj}{Rj}
\DeclareMathOperator{\enc}{Enc}
\DeclareMathOperator{\dec}{Dec}
\DeclareMathOperator{\ids}{IDs}
\DeclareMathOperator{\lgr}{lg}
\DeclareMathOperator{\redu}{red}
\DeclareMathOperator{\head}{hd}
\DeclareMathOperator{\tail}{tl}
\DeclareMathOperator{\hfrag}{hfrag}
\DeclareMathOperator{\hdefrag}{hdefrag}
\DeclareMathOperator{\send}{send}
\DeclareMathOperator{\rec}{receiveAndGroup}

\newcommand\typeT[1]{\text{\ttfamily #1}}
\newcommand{\decryptArgs}[2]{\decrypt_{#1 , \typeT{#2}}}
\newcommand{\cryptArgs}[2]{\crypt_{#1 , \typeT{#2}}}
\newcommand{\projDelta}{\proj_{\delta}}
\newcommand{\selP}{\sel_p}
\newcommand{\decryptCAlpha}{\decryptArgs{\alpha}{c}}
\newcommand{\cryptCAlpha}{\cryptArgs{\alpha}{c}}
\newcommand{\ch}{\typeT{c}}
\newcommand{\chp}{\typeT{c'}}
\newcommand{\groupDelta}{\group_{\delta}}
\newcommand{\fragDelta}{\frag_{\delta}}
\newcommand{\val}{\mathcal{V}}
\newcommand{\cy}[1]{\typeT{c}(#1)}
\newcommand{\dc}[1]{\typeT{c}^{-1}(#1)}
\newcommand{\cip}{\cup \{id\}}
\newcommand{\fold}[3]{\operatorname{fold}_{#1, #2, #3}}
\newcommand{\foldAlphafz}{\fold{\alpha}{f}{z}}
\newcommand{\dilta}{{\delta \cip}}

\newcommand{\guillemets}[1]{\og #1 \fg{}}

\usepackage{listings}
\usepackage{caption}

\lstset{%
	basicstyle=\sffamily,%
	columns=fullflexible,%
	frame=lb,%
	frameround=fftf,%
}%


\begin{document}
\author{Santiago Bautista}
\title{Sémantique et optimisation dans le  langage C2QL}
\date{juin - juillet 2017}

%\titlegraphic{
%	\includegraphics[width=1cm]{logoENS.jpg} \hspace{0.85cm}
%	\includegraphics[width=1.5cm]{logoIMT.jpg} \hspace{0.85cm}
%	\includegraphics[width=1.7cm]{logoINRIA.jpg} \hspace{0.85cm}
%	\includegraphics[width=1.7cm]{logoRennes1.png}}

\titlegraphic{
	\includegraphics[width=2cm]{logoIMT.jpg} \hspace{0.85cm}
	\includegraphics[width=1.3cm]{logoENS.jpg} \hspace{0.85cm}}

\begin{frame}
\titlepage
\end{frame}

\section{Introduction}
\begin{frame}
% De plus en plus d'applications utilisent le cloud
% en manipulant des données personnelles
\end{frame}

\begin{frame}
% Comment protéger la confidentialité
% tout en privilégiant utilisation du nuage et performances?
\end{frame}

\begin{frame}
\tableofcontents
\end{frame}

\begin{frame}{Exemple fil rouge}
% Explorateurs en Australie
\end{frame}


\section{Concepts manipulés et travail antérieur}

\subsection{Confidentialité}
\begin{frame}{Confidentialité}
Un problème de confidentialité est défini par
\begin{itemize}
\item Qu'est-ce qu'on veut protéger?
\item De qui veut-on le protéger?
\end{itemize}
\end{frame}

\begin{frame}{Contraintes de confidentialité}
\end{frame}

\begin{frame}{Hypothèses sur l'attaquant}
\end{frame}

\subsection{Techniques de protection}
\begin{frame}{Chiffrement}
\end{frame}

\begin{frame}{Fragmentation verticale}
\end{frame}

\begin{frame}{Calculs côté utilisateur}
\end{frame}

\subsection{Le langage C2QL et l'algèbre relationnelle}
\begin{frame}{Enjeux d'une application utilisant le nuage}
\end{frame}

\begin{frame}{Enjeux et protection de la confidentialité}
\end{frame}

\begin{frame}{Le langage C2QL}
\end{frame}

\begin{frame}{Relations}
\end{frame}

\begin{frame}{Ensemble des fonctions \uncover<2>{étendu}}
\end{frame}

\subsection{Les lois de commutation}
\begin{frame}{Développer un programme C2QL optimal}
\end{frame}

\section{Contribution}
\subsection{Définir formellement les fonctions de C2QL}
\begin{frame}{Choix à faire}
\end{frame}

\subsection{Prouver la correction des lois}
\begin{frame}{Structure des preuves}
\end{frame}

\begin{frame}{Erreurs corrigées}
\end{frame}

\subsection{Compléter l'ensemble de lois}
\begin{frame}{Critère de complétude}
\end{frame}

\begin{frame}{Exemple}
\end{frame}

\subsection{Une première version de l'optimiseur}
\begin{frame}{Tableau des effets de chaque loi}
\end{frame}

\section{Travail futur}
\begin{frame}
\end{frame}

\section*{Conclusion}
\begin{frame}{Conclusion}
\end{frame}

\end{document}